%\documentclass[10pt,a4paper]{report}
\documentclass{exam}
%encoding
%--------------------------------------
\usepackage[T1]{fontenc}
\usepackage[utf8]{inputenc}
%--------------------------------------
 
%Portuguese-specific commands
%--------------------------------------
\usepackage[portuguese]{babel}
%---------------------------------
\usepackage{graphicx}
\usepackage{comment}
\usepackage{exercise}
\usepackage{enumerate}
\usepackage{amsmath,amssymb,amsthm}
\usepackage[shortlabels]{enumitem}
\usepackage{algorithmic}
\usepackage[ruled,vlined]{algorithm2e}

\usepackage{tikz}
\usepackage{tikz-3dplot}
\usepackage{graphicx}
\usetikzlibrary{%
    decorations.pathreplacing,%
    decorations.pathmorphing%
}
\usetikzlibrary{calc}
\usetikzlibrary{positioning}
%%%%%%%%%%%%%%%%%%%%%%%%%%%%%%%%%%%%%%%%%%%%%%%%%%%%%%%%%%%%%%%%%%%
\newtheorem{theorem}{Theorem}
\newtheorem*{theorem*}{Theorem}
\newtheorem{conjecture}{Conjecture}
\newtheorem{lemma}{Lemma}
\newtheorem{corollary}{Corollary}
\newtheorem{proposition}{Proposition}
\newtheorem{definition}{Definition}
\theoremstyle{definition}
\newtheorem{example}{Example}
\newtheorem{remark}{Observações}

\newcommand\x{\times}
\newcommand\bigzero{\makebox(0,0){\text{\huge0}}}
\newcommand*{\bord}{\multicolumn{1}{c|}{}}
%%%%%%%%%%%%%%%%%%%%%%%%%%%%%%%%%%%%%%%%%%%%%%%%%%%%%%%%%%%%%%%%%%%

%\usepackage[dvipsnames]{xcolor}
%\usepackage{hyperref}

%%%%%%%%%%%%%%%%%%%%%%%%%%%%%%%%%%%%%%%%%%%%%%%%%%%%%%%%%%%%%%%%%%%
% Otra opcion de colores:
%\newcommand\myshade{85}
%\colorlet{mylinkcolor}{violet}
%\colorlet{mycitecolor}{YellowOrange}
%\colorlet{myurlcolor}{Aquamarine}
%\usepackage{hyperref}
%\hypersetup{
%  linkcolor  = mylinkcolor!\myshade!black,
%  citecolor  = mycitecolor!\myshade!black,
%  urlcolor   = myurlcolor!\myshade!black,
%  colorlinks = true,
%}

\usepackage{hyperref}
\usepackage{cleveref}
\hypersetup{
  linkcolor  = blue,
  citecolor  = blue,
  urlcolor   = blue,
  colorlinks = true,
}
%%%%%%%%%%%%%%%%%%%%%%%%%%%%%%%%%%%%%%%%%%%%%%%%%%%%%%%%%%%%%%%%%%%

\usepackage{mathtools}
\usepackage[numbered,framed]{matlab-prettifier}
\usepackage{filecontents}

\newcommand{\vetx}{\mathbf{x}}
%\newcommand{\cos}{\text{cos}}

\DeclarePairedDelimiter\abs{\lvert}{\rvert}%
\DeclarePairedDelimiter\norm{\lVert}{\rVert}%

\usepackage{listings}
\usepackage{color} %red, green, blue, yellow, cyan, magenta, black, white
\definecolor{mygreen}{RGB}{28,172,0} % color values Red, Green, Blue
\definecolor{mylilas}{RGB}{170,55,241}

\lstset{
    language=Octave, %% Troque para PHP, C, Java, etc... bash é o padrão
    breaklines=true,%
    morekeywords={matlab2tikz},
    keywordstyle=\color{blue},%
    morekeywords=[2]{1}, keywordstyle=[2]{\color{black}},
    identifierstyle=\color{black},%
    stringstyle=\color{mylilas},
    commentstyle=\color{mygreen},%
    showstringspaces=false,%without this 
    numbers=left,
    backgroundcolor=\color{gray!10},
    frame=single,
    tabsize=2,
    rulecolor=\color{black!30},
    title=\lstname,
    escapeinside={\%*}{*)},
    breaklines=true,
    breakatwhitespace=true,
    framextopmargin=2pt,
    framexbottommargin=2pt,
    extendedchars=false,
    inputencoding=utf8
}

%%%%%%%%%%%%%%%%%%%%%%%%
%%%%%%%%%%%%%%%%%%%%%%%%
%%%%%%%%%%%%%%%%%%%%%%%%
% Default fixed font does not support bold face
\DeclareFixedFont{\ttb}{T1}{txtt}{bx}{n}{12} % for bold
\DeclareFixedFont{\ttm}{T1}{txtt}{m}{n}{12}  % for normal

% Custom colors
\usepackage{color}
\definecolor{deepblue}{rgb}{0,0,0.5}
\definecolor{deepred}{rgb}{0.6,0,0}
\definecolor{deepgreen}{rgb}{0,0.5,0}

\usepackage{listings}

% Python style for highlighting
\newcommand\pythonstyle{\lstset{
language=Python,
basicstyle=\ttm,
otherkeywords={self},             % Add keywords here
keywordstyle=\ttb\color{deepblue},
emph={MyClass,__init__},          % Custom highlighting
emphstyle=\ttb\color{deepred},    % Custom highlighting style
stringstyle=\color{deepgreen},
frame=tb,                         % Any extra options here
showstringspaces=false            % 
}}


% Python environment
\lstnewenvironment{python}[1][]
{
\pythonstyle
\lstset{#1}
}
{}

% Python for external files
\newcommand\pythonexternal[2][]{{
\pythonstyle
\lstinputlisting[#1]{#2}}}

% Python for inline
\newcommand\pythoninline[1]{{\pythonstyle\lstinline!#1!}}
%%%%%%%%%%%%%%%%%%%%%%%%
%%%%%%%%%%%%%%%%%%%%%%%%
%%%%%%%%%%%%%%%%%%%%%%%%

\oddsidemargin = 8pt
\title{Mini Quiz 2: Conceptos Básicos y Notación.}
\author{Profesor: Rodolfo Anibal Lobo}
\date{Septiembre 2023}

\begin{document}
\maketitle

%\begin{table}[h!]
%\centering
%\begin{tabular}{|l|l|l|l|l|l|l|}
%\hline
% Pregunta & $(1)$ & $(2)$ & $(3)$ & $(4)$  & $(5)$ & Total \\ \hline
% Q1 & $\times$  &$\times$   & $\times$   & $\times$ & $\times$ & $\mathtt{5pts}$  \\ \hline
% Q2 & $\times$ & $\times$  & $\times$   & $\times$   & $\times$  & $\mathtt{5pts}$  \\ \hline
%\end{tabular}
%\end{table}


\section*{Instrucciones}
% \thispagestyle{empty}
% Lea atentamente las preguntas y conteste de manera breve y clara sus respuestas. Recuerde que en negrita expresamos vectores $\mathbf{x}$ fila o coulmna, en cursiva y con sub-indices $x_i$ componentes o escalares, donde $i\in \mathbb{Z}^{+}$, en algunos casos letras mayúsculas pueden representar indistintamente vectores o matrices.

% \begin{enumerate}
% \item Supongamos que tenemos un pokedex programado en $\mathtt{python}$ ,¿Cuáles serían los 3 pasos para poder consultar por Onix?. La clase utilizada para crear el pokedex es la siguiente:

% \begin{python}
% import requests 
% class Pokedex:
%     def __init__(self):
%         # Variables que utilizaremos dentro de toda la clase
%         self.base_url = "https://pokeapi.co/api/v2/pokemon/{}"
%         self.name = None
%         self.type = None
%         self.attack = None
%     def search(self, name: str):
%         # Dando formato al nombre del pokemon en la URL
%         self.name = name.lower()
%         url = self.base_url.format(self.name)
%         # Manejo de excepciones: intento preguntarle a la API
%         try:
%             response = requests.get(url)
%         except: 
%             print("Algo extrano ocurrio")
%         # Observando la respuesta de la API
%         if response.status_code == 200:
%             data = response.json()
%             self.type = data['types'][0]['type']['name']
%             self.attack = data['moves'][0]['move']['name']
%         else:
%             print(f"No pudimos encontrar datos para {self.name}")
%     # Pantalla del pokedex
%     def mostrar_informacion(self):
%         print(f"Nombre: {self.name}")
%         print(f"Tipo: {self.type}")
%         print(f"Ataque: {self.attack}")     
%     \end{python}
% \begin{enumerate}[(1)]
% \item Respuesta:
% \begin{python}
%     my_pokedex = Pokedex()
%     output = Pokedex.search("Pikachu")
%     output.mosgtrar_informacion()
% \end{python}
% \end{enumerate}
%     \item Defina qué es la tasa de aprendizaje o \textit{learning rate}.
%     \begin{enumerate}[(2)]
%     \item \textcolor{cyan}{La tasa de aprendizaje o learning rate corresponde a un hiperparámetro del modelo, pues estebe debe ser ajustado para modificar el tamaño del paso que se dará al momento de realizar la búsqueda por gradiente descendiente. El parámetro multiplica la derivada parcial de la función de pérdida en relación a los parámetros por optimizar. Si el parámetro es muy grande podemos tener problemas para encontrar el mínimo local y diverger, en el caso contrario, cuando es muy pequeño podemos tardar muchísimo tiempo en encontrar un mínimo pues avanzamos lentamente en dirección a la solución.} 
%     \end{enumerate}
\thispagestyle{empty}
Lea atentamente las preguntas y conteste de manera breve y clara sus respuestas. Recuerde que en negrita expresamos vectores $\mathbf{x}$ fila o coulmna, en cursiva y con sub-indices $x_i$ componentes o escalares, donde $i\in \mathbb{Z}^{+}$, en algunos casos letras mayúsculas pueden representar indistintamente vectores o matrices. \textbf{Elija 3 preguntas}.

\begin{enumerate}
\item ¿Qué es una regresión lineal?, defina de la manera más formal posible y de un ejemplo de aplicación.
\textcolor{cyan}{EN particular para nuestros ejemplos , una regresión lineal es un método que permite ajustar parámetros de un modelo lineal para poder ajustar una recta a un conjunto de datos, o en palabras simples ajustar un modelo para predecir la tendencia del fenómeno en estudio a través de una recta (de forma general hablamos de hiperplanos). La finalidad, es interpolar y extrapolar valores reales a lo largo del dominio de estudio. Posibles ejemplos de aplicación: predecir valores o precios de diferentes objetos que dependen de alguna variable, completar valores en una serie temporal (como una voz o sonido).} 

\item Escriba a un costado de cada $\mathtt{print}$ lo que imprimirá la función en cada caso:
    \begin{python}
    import numpy as np
     x = [1,2,3]
     y = [4,5,6]
     print(x+y)
    
     x_vec = np.array([1,2,3])
     y_vec = np.array([4,5,6])
     print(x_vec+y_vec)
 
    \end{python}
    \begin{itemize}
        \item \textcolor{cyan}{$\mathtt{[1,2,3,4,5,6]}$.} 
        \item  \textcolor{cyan}{$\mathtt{[5,7,9]}$.} 
    \end{itemize}
\item Supongamos que tenemos un conjunto de audios $A = \{\mathbf{a}_1, \mathbf{a}_2,\dots,\mathbf{a}_n\}$ donde cada $\mathbf{a}_i$ con $i = 1,\dots, n$ es un vector con $m$ muestras, explícitamente $\mathbf{a}_i = [x_0,x_1,x_2,\dots,x_m]$. Si queremos entrenar un modelo para hallar los parámetros $\theta$ en la ecuación:
\begin{equation}
    D \theta = Y + \epsilon
\end{equation}
donde $\epsilon$ es una matriz de error, $D$ es una matriz que tiene por filas las primeras $p<m$ muestras de un audio e $Y$ una matriz con las últimas $m-p$ muestras. Entonces, ¿este es un problema de regresión o clasificación?, ¿por qué?
\begin{itemize}
    \item \textcolor{cyan}{Es un problema de regresión lineal. Mismo ejemplo visto en clases para completar voces de audios. En este caso la diferencia es que es un problema multilineal, donde $\Theta$ e $Y$ son matrices, sin embargo, el problema puede ser resuelto via álgebra lineal a través de ecuaciones normales, en este caso $Y$ es el objetivo y $D$ es la matriz de datos con las entradas. No es un problema de clasificación pues $Y$ no representa variables categóricas como nombres o clases, en cambio, los audios fueron divididos en dos partes con el objetivo de completar la serie temporal incompleta en los inputs a través de un modelo multilineal.} 
\end{itemize}

    \item Dadas las siguientes ecuaciones normales:
    \begin{equation}
      \Theta = (X^{T}X)^{-1}X^{T}\mathbf{y}
    \end{equation}
    Si $\mathbf{x} \in \mathbb{R}^{1\times N}$ es una fila de la matriz $X$, es decir $\mathbf{x}=[x_1,x_2,\dots,x_N]$ nuestros datos, para los cuales tenemos 260 ejemplos. Además sea $\theta\in \mathbb{R}^{6\times 1}$ e $y_i \in \mathbb{R}$. ¿Cuáles son las dimensiones de las siguientes matrices-vectores?. Exprese las dimensiones explícitamente utilizando números enteros positivos.
    \begin{enumerate}
        \item \textcolor{cyan}{$X$: $\left(260\times 6\right)$ }
        \item \textcolor{cyan}{$X^{T}X$: $\left(6\times 6\right)$ }
        \item\textcolor{cyan}{ $(X^{T}X)^{-1}$: $\left(6\times 6\right)$ }
        \item \textcolor{cyan}{$X^{T}\mathbf{y}$: $\left(6\times 1\right)$ }
        \item \textcolor{cyan}{$\mathbf{x}$: $\left(1\times 6\right)$ }
        \item \textcolor{cyan}{$(X^{T}X)^{-1}X^{T}$: $\left(6\times 260\right)$ }
        \item \textcolor{cyan}{$\mathbf{y}$: $\left(260\times 1\right)$ }
        \item \textcolor{cyan}{$(X^{T}X)^{-1}X^{T}\mathbf{y}$:$\left(6\times1\right)$}
    \end{enumerate}
\end{enumerate}
\end{document}
